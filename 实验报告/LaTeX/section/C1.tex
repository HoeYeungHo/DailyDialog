\section{实验背景}

科学技术的进步贯穿了人类发展的方方面面。它不仅仅是技术和知识的积累,更是一种持续革新的过程,对人类社会产生了深远的影响。随着时间的推移,人们对技术和科学的理解逐渐深化,并在不断探索中催生了计算机科学的诞生和蓬勃发展。

20世纪下半叶至21世纪初期,计算机技术经历了翻天覆地的变革。从最初的巨型机器,体积庞大、运算速度缓慢,到如今的个人电脑和智能移动设备,计算机的性能和普及程度实现了巨大的跨越。这种变化不仅改变了人们的工作和生活方式,也影响了社会的方方面面。

计算机的进步推动了信息时代的来临。在互联网的浪潮下,信息传播变得更快捷、全球联系更加紧密。个人电脑、智能手机等智能设备成为人们日常不可或缺的工具,社交、工作、娱乐等各个领域都得到了极大的改善和便利。这种计算机技术的迅速普及也为其他领域的发展铺平了道路,其中之一就是自然语言处理(Natural LanguageProcessing,NLP)。NLP致力于使计算机理解和处理人类语言,其发展也是计算机技术进步的一个重要方向。随着NLP技术的不断进步,语音识别、文本理解、自动翻译等领域取得了显著的进展,为人机交互提供了更加智能化的解决方案。

在NLP领域,预训练模型的出现更是带来了革命性的变化。这些模型利用大规模的文本数据进行预先训练,使得计算机能够更好地理解语言的结构和语境,提高了在各种NLP任务上的性能表现。BERT(Bidirectional Encoder Representations from Transformers)\cite{BERT}和GPT(Generative Pre-trained Transformer)\cite{GPT}等预训练模型的问世,为自然语言处理领域带来了新的里程碑。

情感识别作为NLP领域的一个重要分支,旨在识别和理解文本中的情感和情绪。随着人们在社交媒体上的大量交流,情感识别技术越来越重要。预训练模型的运用使得情感识别更加精准和高效,为企业的市场营销、舆情分析等提供了有力支持。

本实验基于预训练编码模型BERT,对情感识别领域中的对话情感识别进行探讨。全文共4章,剩余章节安排如下:

第\ref{sec:2}章:\nameref{sec:2}。主要介绍此次实验用到的各种神经网络,包括循环神经网络、分词器、Transformer架构\cite{transformer}和BERT模型。

第\ref{sec:3}章:\nameref{sec:3}。主要介绍此次实验的数据集以及所构建的模型。并且对实验结果进行了详细地分析。

第\ref{sec:4}章:\nameref{sec:4}。总结此次实验的过程以及收获。









